% Options for packages loaded elsewhere
% Options for packages loaded elsewhere
\PassOptionsToPackage{unicode}{hyperref}
\PassOptionsToPackage{hyphens}{url}
\PassOptionsToPackage{dvipsnames,svgnames,x11names}{xcolor}
%
\documentclass[
  portuguese,
  letterpaper,
  DIV=11,
  numbers=noendperiod]{scrartcl}
\usepackage{xcolor}
\usepackage{amsmath,amssymb}
\setcounter{secnumdepth}{-\maxdimen} % remove section numbering
\usepackage{iftex}
\ifPDFTeX
  \usepackage[T1]{fontenc}
  \usepackage[utf8]{inputenc}
  \usepackage{textcomp} % provide euro and other symbols
\else % if luatex or xetex
  \usepackage{unicode-math} % this also loads fontspec
  \defaultfontfeatures{Scale=MatchLowercase}
  \defaultfontfeatures[\rmfamily]{Ligatures=TeX,Scale=1}
\fi
\usepackage{lmodern}
\ifPDFTeX\else
  % xetex/luatex font selection
\fi
% Use upquote if available, for straight quotes in verbatim environments
\IfFileExists{upquote.sty}{\usepackage{upquote}}{}
\IfFileExists{microtype.sty}{% use microtype if available
  \usepackage[]{microtype}
  \UseMicrotypeSet[protrusion]{basicmath} % disable protrusion for tt fonts
}{}
\makeatletter
\@ifundefined{KOMAClassName}{% if non-KOMA class
  \IfFileExists{parskip.sty}{%
    \usepackage{parskip}
  }{% else
    \setlength{\parindent}{0pt}
    \setlength{\parskip}{6pt plus 2pt minus 1pt}}
}{% if KOMA class
  \KOMAoptions{parskip=half}}
\makeatother
% Make \paragraph and \subparagraph free-standing
\makeatletter
\ifx\paragraph\undefined\else
  \let\oldparagraph\paragraph
  \renewcommand{\paragraph}{
    \@ifstar
      \xxxParagraphStar
      \xxxParagraphNoStar
  }
  \newcommand{\xxxParagraphStar}[1]{\oldparagraph*{#1}\mbox{}}
  \newcommand{\xxxParagraphNoStar}[1]{\oldparagraph{#1}\mbox{}}
\fi
\ifx\subparagraph\undefined\else
  \let\oldsubparagraph\subparagraph
  \renewcommand{\subparagraph}{
    \@ifstar
      \xxxSubParagraphStar
      \xxxSubParagraphNoStar
  }
  \newcommand{\xxxSubParagraphStar}[1]{\oldsubparagraph*{#1}\mbox{}}
  \newcommand{\xxxSubParagraphNoStar}[1]{\oldsubparagraph{#1}\mbox{}}
\fi
\makeatother

\usepackage{color}
\usepackage{fancyvrb}
\newcommand{\VerbBar}{|}
\newcommand{\VERB}{\Verb[commandchars=\\\{\}]}
\DefineVerbatimEnvironment{Highlighting}{Verbatim}{commandchars=\\\{\}}
% Add ',fontsize=\small' for more characters per line
\usepackage{framed}
\definecolor{shadecolor}{RGB}{241,243,245}
\newenvironment{Shaded}{\begin{snugshade}}{\end{snugshade}}
\newcommand{\AlertTok}[1]{\textcolor[rgb]{0.68,0.00,0.00}{#1}}
\newcommand{\AnnotationTok}[1]{\textcolor[rgb]{0.37,0.37,0.37}{#1}}
\newcommand{\AttributeTok}[1]{\textcolor[rgb]{0.40,0.45,0.13}{#1}}
\newcommand{\BaseNTok}[1]{\textcolor[rgb]{0.68,0.00,0.00}{#1}}
\newcommand{\BuiltInTok}[1]{\textcolor[rgb]{0.00,0.23,0.31}{#1}}
\newcommand{\CharTok}[1]{\textcolor[rgb]{0.13,0.47,0.30}{#1}}
\newcommand{\CommentTok}[1]{\textcolor[rgb]{0.37,0.37,0.37}{#1}}
\newcommand{\CommentVarTok}[1]{\textcolor[rgb]{0.37,0.37,0.37}{\textit{#1}}}
\newcommand{\ConstantTok}[1]{\textcolor[rgb]{0.56,0.35,0.01}{#1}}
\newcommand{\ControlFlowTok}[1]{\textcolor[rgb]{0.00,0.23,0.31}{\textbf{#1}}}
\newcommand{\DataTypeTok}[1]{\textcolor[rgb]{0.68,0.00,0.00}{#1}}
\newcommand{\DecValTok}[1]{\textcolor[rgb]{0.68,0.00,0.00}{#1}}
\newcommand{\DocumentationTok}[1]{\textcolor[rgb]{0.37,0.37,0.37}{\textit{#1}}}
\newcommand{\ErrorTok}[1]{\textcolor[rgb]{0.68,0.00,0.00}{#1}}
\newcommand{\ExtensionTok}[1]{\textcolor[rgb]{0.00,0.23,0.31}{#1}}
\newcommand{\FloatTok}[1]{\textcolor[rgb]{0.68,0.00,0.00}{#1}}
\newcommand{\FunctionTok}[1]{\textcolor[rgb]{0.28,0.35,0.67}{#1}}
\newcommand{\ImportTok}[1]{\textcolor[rgb]{0.00,0.46,0.62}{#1}}
\newcommand{\InformationTok}[1]{\textcolor[rgb]{0.37,0.37,0.37}{#1}}
\newcommand{\KeywordTok}[1]{\textcolor[rgb]{0.00,0.23,0.31}{\textbf{#1}}}
\newcommand{\NormalTok}[1]{\textcolor[rgb]{0.00,0.23,0.31}{#1}}
\newcommand{\OperatorTok}[1]{\textcolor[rgb]{0.37,0.37,0.37}{#1}}
\newcommand{\OtherTok}[1]{\textcolor[rgb]{0.00,0.23,0.31}{#1}}
\newcommand{\PreprocessorTok}[1]{\textcolor[rgb]{0.68,0.00,0.00}{#1}}
\newcommand{\RegionMarkerTok}[1]{\textcolor[rgb]{0.00,0.23,0.31}{#1}}
\newcommand{\SpecialCharTok}[1]{\textcolor[rgb]{0.37,0.37,0.37}{#1}}
\newcommand{\SpecialStringTok}[1]{\textcolor[rgb]{0.13,0.47,0.30}{#1}}
\newcommand{\StringTok}[1]{\textcolor[rgb]{0.13,0.47,0.30}{#1}}
\newcommand{\VariableTok}[1]{\textcolor[rgb]{0.07,0.07,0.07}{#1}}
\newcommand{\VerbatimStringTok}[1]{\textcolor[rgb]{0.13,0.47,0.30}{#1}}
\newcommand{\WarningTok}[1]{\textcolor[rgb]{0.37,0.37,0.37}{\textit{#1}}}

\usepackage{longtable,booktabs,array}
\usepackage{calc} % for calculating minipage widths
% Correct order of tables after \paragraph or \subparagraph
\usepackage{etoolbox}
\makeatletter
\patchcmd\longtable{\par}{\if@noskipsec\mbox{}\fi\par}{}{}
\makeatother
% Allow footnotes in longtable head/foot
\IfFileExists{footnotehyper.sty}{\usepackage{footnotehyper}}{\usepackage{footnote}}
\makesavenoteenv{longtable}
\usepackage{graphicx}
\makeatletter
\newsavebox\pandoc@box
\newcommand*\pandocbounded[1]{% scales image to fit in text height/width
  \sbox\pandoc@box{#1}%
  \Gscale@div\@tempa{\textheight}{\dimexpr\ht\pandoc@box+\dp\pandoc@box\relax}%
  \Gscale@div\@tempb{\linewidth}{\wd\pandoc@box}%
  \ifdim\@tempb\p@<\@tempa\p@\let\@tempa\@tempb\fi% select the smaller of both
  \ifdim\@tempa\p@<\p@\scalebox{\@tempa}{\usebox\pandoc@box}%
  \else\usebox{\pandoc@box}%
  \fi%
}
% Set default figure placement to htbp
\def\fps@figure{htbp}
\makeatother


% definitions for citeproc citations
\NewDocumentCommand\citeproctext{}{}
\NewDocumentCommand\citeproc{mm}{%
  \begingroup\def\citeproctext{#2}\cite{#1}\endgroup}
\makeatletter
 % allow citations to break across lines
 \let\@cite@ofmt\@firstofone
 % avoid brackets around text for \cite:
 \def\@biblabel#1{}
 \def\@cite#1#2{{#1\if@tempswa , #2\fi}}
\makeatother
\newlength{\cslhangindent}
\setlength{\cslhangindent}{1.5em}
\newlength{\csllabelwidth}
\setlength{\csllabelwidth}{3em}
\newenvironment{CSLReferences}[2] % #1 hanging-indent, #2 entry-spacing
 {\begin{list}{}{%
  \setlength{\itemindent}{0pt}
  \setlength{\leftmargin}{0pt}
  \setlength{\parsep}{0pt}
  % turn on hanging indent if param 1 is 1
  \ifodd #1
   \setlength{\leftmargin}{\cslhangindent}
   \setlength{\itemindent}{-1\cslhangindent}
  \fi
  % set entry spacing
  \setlength{\itemsep}{#2\baselineskip}}}
 {\end{list}}
\usepackage{calc}
\newcommand{\CSLBlock}[1]{\hfill\break\parbox[t]{\linewidth}{\strut\ignorespaces#1\strut}}
\newcommand{\CSLLeftMargin}[1]{\parbox[t]{\csllabelwidth}{\strut#1\strut}}
\newcommand{\CSLRightInline}[1]{\parbox[t]{\linewidth - \csllabelwidth}{\strut#1\strut}}
\newcommand{\CSLIndent}[1]{\hspace{\cslhangindent}#1}

\ifLuaTeX
\usepackage[bidi=basic]{babel}
\else
\usepackage[bidi=default]{babel}
\fi
% get rid of language-specific shorthands (see #6817):
\let\LanguageShortHands\languageshorthands
\def\languageshorthands#1{}


\setlength{\emergencystretch}{3em} % prevent overfull lines

\providecommand{\tightlist}{%
  \setlength{\itemsep}{0pt}\setlength{\parskip}{0pt}}



 


\KOMAoption{captions}{tableheading}
\makeatletter
\@ifpackageloaded{caption}{}{\usepackage{caption}}
\AtBeginDocument{%
\ifdefined\contentsname
  \renewcommand*\contentsname{Índice}
\else
  \newcommand\contentsname{Índice}
\fi
\ifdefined\listfigurename
  \renewcommand*\listfigurename{Lista de Figuras}
\else
  \newcommand\listfigurename{Lista de Figuras}
\fi
\ifdefined\listtablename
  \renewcommand*\listtablename{Lista de Tabelas}
\else
  \newcommand\listtablename{Lista de Tabelas}
\fi
\ifdefined\figurename
  \renewcommand*\figurename{Figura}
\else
  \newcommand\figurename{Figura}
\fi
\ifdefined\tablename
  \renewcommand*\tablename{Tabela}
\else
  \newcommand\tablename{Tabela}
\fi
}
\@ifpackageloaded{float}{}{\usepackage{float}}
\floatstyle{ruled}
\@ifundefined{c@chapter}{\newfloat{codelisting}{h}{lop}}{\newfloat{codelisting}{h}{lop}[chapter]}
\floatname{codelisting}{Listagem}
\newcommand*\listoflistings{\listof{codelisting}{Lista de Listagens}}
\makeatother
\makeatletter
\makeatother
\makeatletter
\@ifpackageloaded{caption}{}{\usepackage{caption}}
\@ifpackageloaded{subcaption}{}{\usepackage{subcaption}}
\makeatother
\usepackage{bookmark}
\IfFileExists{xurl.sty}{\usepackage{xurl}}{} % add URL line breaks if available
\urlstyle{same}
\hypersetup{
  pdftitle={Trabalho Final - Elementos de Programação para Estatística},
  pdfauthor={Antonio Paulo Steffen Neto; Ingrid Giacomeli; Jhullya da Rosa Shalders; Pedro Henrique D'Andrea},
  pdflang={pt},
  colorlinks=true,
  linkcolor={blue},
  filecolor={Maroon},
  citecolor={Blue},
  urlcolor={Blue},
  pdfcreator={LaTeX via pandoc}}


\title{Trabalho Final - Elementos de Programação para Estatística}
\usepackage{etoolbox}
\makeatletter
\providecommand{\subtitle}[1]{% add subtitle to \maketitle
  \apptocmd{\@title}{\par {\large #1 \par}}{}{}
}
\makeatother
\subtitle{Prova 5}
\author{Antonio Paulo Steffen Neto \and Ingrid Giacomeli \and Jhullya da
Rosa Shalders \and Pedro Henrique D'Andrea}
\date{2025-07-12}
\begin{document}
\maketitle


\subsection{Introdução}\label{sec-01}

Este trabalho se propõe a ser a parte de conclusão do Módulo 5 da
discilpina CE302 - Elementos de Programação para Estatística. Para tal,
ele consiste em uma análise de um conjunto de dados disponibilizado
pelos professores responsáveis e subsequente desenvolvimento de um
aplicativo em Shiny para visualização dos resultados encontrados. Assim,
este trabalho apresenta a parte técnica e teórica do que foi realizado
pelo grupo. A parte da apresentação em um aplicativo em Shiny, por outro
lado, será feita por meio de um vídeo de apresentação. Nesta introdução
apresentamos, então, o conjunto de dados escolhido pelo grupo e, também,
como a parte escrita do trabalho está estruturada.

Sobre a escolha do conjuntos de dados disponibilizados, o grupo optou
pelo conjunto da dados chamado \emph{Allrecipes}. Neste constam dados
retirados do site \href{https://www.allrecipes.com}{Allrecipes} sobre
receitas, seus países de origem e autores, informações nutricionais,
tempo de preparo, avaliações e vezes em que foram reproduzidas. Assim
sendo, todas as informações a serem referenciadas no texto têm como
fonte a base de dados da própria plataforma {«Allrecipes {\textbar}
{Recipes}, {How}-{Tos}, {Videos} and {More}»} (\emph{{[}S.d.{]}}).

Como definido pelas instruções dadas pela proposta de trabalho, a
formatação deste segue a ordem: Introdução; Materiais e Métodos;
Resultados e Discussão; Conclusão e Referências. Muitas das partes
integrantes da ordem originalmente indicada, naturalmente, apresentam
suas próprias subseções, assim sendo, esta introdução conta com duas
subseções: Contextualização e Definição do Problema e Objetivos da
Análise. A parte de Materiais e Métodos, por sua vez, tem como
sbdivisões: Descrição do Conjunto de Dados, Dicionário de Variáveis,
Tratamento de Dados e Técnicas Estatísticas e Computacionais. Na
sequência, são apresentados os Resultados e Discussão, divididos de
acordo com temas de análise, com suas respectivas análises e
interpretações como será, posteriormente, fácil de analisar com o
aplicativo Shiny feito em conjunto com esta parte escrita. Por fim, a
Conclusão e as Referências Bibliográficas não apresentam quaisquer
divisões.

\subsubsection{Contextualização e Definição do
Problema}\label{contextualizauxe7uxe3o-e-definiuxe7uxe3o-do-problema}

A alimentação é a base de toda a cultura humana, o momento de comunhão
em diferentes culturas e religiões. Isso estende-se desde o cômodo onde
as pessoas se alimentam até a como grandes festividades por todo o globo
costumam ter alguma refeição como parte central das celebrações. Mais
recentemente, o destaque adquirido por grandes chefes de cosinha e por
produtos culturais como reality show e séries sobre culinária são
sintomas deste fenômeno.

Mais recentemente, contudo, a relação das pessoas com a comida foi muito
modificada e tornada mais complexa. Em tempos de internet, fast food e
fácil acesso a ingredientes de todo o mundo, sites como o
\emph{Allrecipes} se tornaram um meio impressindível de compartilhamento
de hábitos alimentares e, para pesquisadores, uma fonte preciosa de
dados para entender como as pessoas se alimentam. Mas sobre o que se
trata o site em questão? O \emph{Allrecipes} consiste em um site de
compartilhamento de receitas. Nele, usuários de todo o mundo podem
postar suas receitas para que outras pessoas possam prová-las,
acompanhar as avaliações dos outros usuários e, também, acessar receitas
de outros usuários, igualmente prová-las e avaliá-las. O site
\emph{Allrecipes}, portanto, e já olhando para os dados de interesse na
presente análise, nos traz dados a respeito da origem das receitas e por
quem foram postadas. Além disso, também inclui as informações acerca de
tempos de preparo e, também, de questões nutricionais.

A análise de dados não estruturados provenientes da plataforma online
\emph{Allrecipes} insere-se na emergente área dGastronomia
Computacional, que tem por objetivo o emprego de técnicas
\emph{data-driven} para o estudo e a análise de comidas. Esse campo
busca quantificar e modelar fenômenos culinários e de consumo alimentar.
Neste contexto, o volume e a variedade dos dados de receitas (incluindo
as avaliações, informações nutricionais e origem geográfica) constituem
um conjunto de dados de nicho, que seria ideal para a aplicação de
métodos de aprendizado de máquina e modelagem estatística, caso os
autores deste trabalho estivessem mais adiante no curso de Estatística.

\subsubsection{Objetivos da Análise}\label{objetivos-da-anuxe1lise}

Dado o conjunto de dados escolhido, portanto, é possível fazer uma
análise que cubra de maneira completa todas as variáveis a disposição,
e, deste modo, explorar e documentar as dinâmicas relacionadas à
popularidade de receitas publicadas online. Assim, abrem-se diversas
direções de investigação, pois tanto as informações nutricionais quanto
as informações acerca dos ingredientes podem ser entrelaçadas e
apresentam grande complexidade. Além destas, há as informações acerca
dos países de origem das receitas, suas respectivas avaliações e, também
tempos de preparo, datas em que foram postadas e quantas vezes foram
preparadas.

O objetivo principal da análise, portanto, é o de entender quais as
dinâmicas por trás da popularidade de receitas publicadas online na
plataforma \emph{Allrecipes}. Assim sendo, a análise é conduzida em
partes e em diferentes frentes que englobam os diferentes tipos de
variáveis à disposição nos conjuntos de dados disponíveis olhando para
questões culturais, nutricionais, temporais e de popularidade das
receitas publicadas. As divisões, como foram feitas, se apresentam da
seguinte maneira:

\begin{enumerate}
\def\labelenumi{\arabic{enumi}.}
\tightlist
\item
  Panorama Geral: Caracterizar o perfil nutricional predominante das
  receitas mais populares. Isto envolverá a limpeza e harmonização das
  variáveis contínuas (calorias, gordura, proteína, sódio) e o uso de
  métricas de dispersão para comparar a centralidade desses atributos em
  relação ao sucesso de reprodução.
\item
  Países e Culinárias: O objetivo é identificar visualmente (via
  gráficos de dispersão) se existe uma correlação descritiva entre a
  complexidade do preparo e as avaliações médias.
\item
  As avaliações de acordo com os ingredientes: Dividir e listar os
  ingredientes apresentados e relacioná-los com as avaliações médias e
  totais dis tribuídas por cada receita.
\item
  Análise Nutricional: Partindo das informações acerca de calorias,
  proteínas, gordura e carboidratos à disposição, serão traçados perfis
  a serem relacionados, também, com as avaliações das receitas.
\item
  As avaliações de acordo com tempos de preparo: Seguindo a mesma lógica
  dos itens anteriores, as diferentes métricas de tempo de preparo das
  receitas serão analisadas e avaliadas de acordo com a existência, ou
  não, de relação com as avaliações das receitas.
\item
  Popularidades das cozinhas e países: Relacionando tudo o que foi
  mostrado anteriormente, serão utilizadas medidas de dispersão para
  relacionar os itens anteriores com as informações de origes de cada
  prato e culinária.
\end{enumerate}

\subsection{Materiais e Métodos}\label{sec-02}

Teremos como principal material de apoio, suplemntar aos materiais da
disciplina, o livro \emph{R for data science: import, tidy, transform,
visualize, and model data}, de autoria de Wickham; Grolemund (2017).
Também serão utilizados para análise de observações o que é desenvolvido
por Agresti (2019).

Além de ambos, também serão utilizados materiais de Trattner; Elsweiler;
Howard (2017) e Hussain et al. (2025), que já desenvolveram análises
semelhantes à que este trabalho se propõe. Além disso, também será
utilizada Silva Da Costa; Amorim (2021)

\subsubsection{Descrição do Conjunto de
Dados}\label{descriuxe7uxe3o-do-conjunto-de-dados}

Para fazer a descrição dos conjuntos de dados fornecidos, começamos
carregando os dados e utilizando a função \texttt{str()} para vermos as
descrições estruturais de ambos os conjuntos.

\begin{Shaded}
\begin{Highlighting}[]
\FunctionTok{library}\NormalTok{(tidyverse)}
\end{Highlighting}
\end{Shaded}

\begin{verbatim}
-- Attaching core tidyverse packages ------------------------ tidyverse 2.0.0 --
v dplyr     1.1.4     v readr     2.1.5
v forcats   1.0.0     v stringr   1.5.1
v ggplot2   4.0.0     v tibble    3.3.0
v lubridate 1.9.4     v tidyr     1.3.1
v purrr     1.1.0     
-- Conflicts ------------------------------------------ tidyverse_conflicts() --
x dplyr::filter() masks stats::filter()
x dplyr::lag()    masks stats::lag()
i Use the conflicted package (<http://conflicted.r-lib.org/>) to force all conflicts to become errors
\end{verbatim}

\begin{Shaded}
\begin{Highlighting}[]
\FunctionTok{library}\NormalTok{(tidytuesdayR)}
\end{Highlighting}
\end{Shaded}

\begin{verbatim}
Warning: pacote 'tidytuesdayR' foi compilado no R versão 4.5.2
\end{verbatim}

\begin{Shaded}
\begin{Highlighting}[]
\CommentTok{\# Carregamento dos dados originais}
\NormalTok{all\_recipes }\OtherTok{\textless{}{-}}\NormalTok{ readr}\SpecialCharTok{::}\FunctionTok{read\_csv}\NormalTok{(}\StringTok{\textquotesingle{}https://raw.githubusercontent.com/rfordatascience/tidytuesday/main/data/2025/2025{-}09{-}16/all\_recipes.csv\textquotesingle{}}\NormalTok{, }\AttributeTok{show\_col\_types =} \ConstantTok{FALSE}\NormalTok{)}

\NormalTok{cuisines }\OtherTok{\textless{}{-}}\NormalTok{ readr}\SpecialCharTok{::}\FunctionTok{read\_csv}\NormalTok{(}\StringTok{\textquotesingle{}https://raw.githubusercontent.com/rfordatascience/tidytuesday/main/data/2025/2025{-}09{-}16/cuisines.csv\textquotesingle{}}\NormalTok{, }\AttributeTok{show\_col\_types =} \ConstantTok{FALSE}\NormalTok{)}

\FunctionTok{str}\NormalTok{(all\_recipes)}
\end{Highlighting}
\end{Shaded}

\begin{verbatim}
spc_tbl_ [14,426 x 16] (S3: spec_tbl_df/tbl_df/tbl/data.frame)
 $ name          : chr [1:14426] "Chewy Whole Wheat Peanut Butter Brownies" "Pumpkin Pie Eggnog" "Eggs Poached in Tomato Sauce" "Minestrone Casserole" ...
 $ url           : chr [1:14426] "https://www.allrecipes.com/recipe/140717/chewy-whole-wheat-peanut-butter-brownies/" "https://www.allrecipes.com/recipe/269204/pumpkin-pie-eggnog/" "https://www.allrecipes.com/recipe/238054/eggs-poached-in-tomato-sauce/" "https://www.allrecipes.com/minestrone-casserole-recipe-8765618" ...
 $ author        : chr [1:14426] "DMOMMY" "Bobbie Susan" "Bren" "Sarah Brekke" ...
 $ date_published: Date[1:14426], format: "2020-06-18" "2022-09-26" ...
 $ ingredients   : chr [1:14426] "⅓ cup margarine, softened, ⅔ cup white sugar, ½ cup packed brown sugar, 2  eggs, 1 cup peanut butter, ½ teaspoo"| __truncated__ "12  egg yolks, 2 cups heavy whipping cream, ½ teaspoon vanilla extract, 1 (15 ounce) can pumpkin puree, ½ cup l"| __truncated__ "2 tablespoons olive oil, or to taste, ½  onion, finely chopped, 2 cloves garlic, finely chopped, 8 cups tomato "| __truncated__ "4 cups dried mafalda pasta (mini lasagna noodles), 2 tablespoons olive oil, 2 carrots, sliced, 2 stalks celery,"| __truncated__ ...
 $ calories      : num [1:14426] 222 477 354 356 366 709 466 782 355 395 ...
 $ fat           : num [1:14426] 13 31 18 9 22 47 27 61 15 12 ...
 $ carbs         : num [1:14426] 24 43 32 53 23 31 1 19 33 33 ...
 $ protein       : num [1:14426] 6 8 20 19 19 37 52 40 23 37 ...
 $ avg_rating    : num [1:14426] 4.4 5 4.8 4.3 4.7 4.2 4.4 4.6 4.7 4.7 ...
 $ total_ratings : num [1:14426] 47 1 4 14 84 5 648 347 129 195 ...
 $ reviews       : num [1:14426] 36 1 4 13 67 3 468 259 102 153 ...
 $ prep_time     : num [1:14426] 20 10 10 20 30 45 15 10 30 20 ...
 $ cook_time     : num [1:14426] 35 5 75 40 95 80 45 190 480 55 ...
 $ total_time    : num [1:14426] 55 495 85 60 125 155 60 200 510 75 ...
 $ servings      : num [1:14426] 16 8 4 8 8 12 6 8 12 6 ...
 - attr(*, "spec")=
  .. cols(
  ..   name = col_character(),
  ..   url = col_character(),
  ..   author = col_character(),
  ..   date_published = col_date(format = ""),
  ..   ingredients = col_character(),
  ..   calories = col_double(),
  ..   fat = col_double(),
  ..   carbs = col_double(),
  ..   protein = col_double(),
  ..   avg_rating = col_double(),
  ..   total_ratings = col_double(),
  ..   reviews = col_double(),
  ..   prep_time = col_double(),
  ..   cook_time = col_double(),
  ..   total_time = col_double(),
  ..   servings = col_double()
  .. )
 - attr(*, "problems")=<externalptr> 
\end{verbatim}

\begin{Shaded}
\begin{Highlighting}[]
\FunctionTok{str}\NormalTok{(cuisines)}
\end{Highlighting}
\end{Shaded}

\begin{verbatim}
spc_tbl_ [2,218 x 17] (S3: spec_tbl_df/tbl_df/tbl/data.frame)
 $ name          : chr [1:2218] "Saganaki (Flaming Greek Cheese)" "Coney Island Knishes" "Diana's Hawaiian Bread Rolls" "Chilean Pebre" ...
 $ country       : chr [1:2218] "Greek" "Jewish" "Australian and New Zealander" "Chilean" ...
 $ url           : chr [1:2218] "https://www.allrecipes.com/recipe/263750/flaming-greek-cheese-saganaki/" "https://www.allrecipes.com/recipe/272334/coney-island-knishes/" "https://www.allrecipes.com/recipe/22797/dianas-hawaiian-bread-rolls/" "https://www.allrecipes.com/recipe/273763/chilean-pebre/" ...
 $ author        : chr [1:2218] "John Mitzewich" "John Mitzewich" "CHIPPENDALE" "Heidi" ...
 $ date_published: Date[1:2218], format: "2024-02-07" "2024-11-26" ...
 $ ingredients   : chr [1:2218] "1 (4 ounce) package kasseri cheese, 1 tablespoon water, or as needed, ¼ cup all-purpose flour, or as needed, 1 "| __truncated__ "2 ¾ cups all-purpose flour, or more as needed, 1 teaspoon kosher salt, 1 teaspoon baking powder, ½ cup canola o"| __truncated__ "1 ½ cups warm water (110 degrees F/45 degrees C), 1  egg, 1 teaspoon salt, 1 teaspoon vanilla extract, 1 teaspo"| __truncated__ "½ cup chopped cilantro, ¼ cup olive oil, ¼ cup red wine vinegar, 1  tomato, chopped, 1 small onion, chopped, 1 "| __truncated__ ...
 $ calories      : num [1:2218] 391 301 64 106 449 958 378 90 157 322 ...
 $ fat           : num [1:2218] 25 17 3 9 23 24 10 5 6 16 ...
 $ carbs         : num [1:2218] 15 31 9 7 58 144 59 10 25 39 ...
 $ protein       : num [1:2218] 16 7 1 1 7 46 14 1 2 7 ...
 $ avg_rating    : num [1:2218] 4.8 4.6 4.3 5 3.8 4.4 4.3 NA 4.6 5 ...
 $ total_ratings : num [1:2218] 25 10 126 1 13 40 3 NA 65 2 ...
 $ reviews       : num [1:2218] 22 9 104 1 11 32 3 NA 55 2 ...
 $ prep_time     : num [1:2218] 10 30 20 10 30 30 30 40 0 5 ...
 $ cook_time     : num [1:2218] 5 75 15 0 15 165 75 30 0 5 ...
 $ total_time    : num [1:2218] 15 180 180 10 45 675 585 155 0 10 ...
 $ servings      : num [1:2218] 2 16 12 6 15 6 6 84 24 1 ...
 - attr(*, "spec")=
  .. cols(
  ..   name = col_character(),
  ..   country = col_character(),
  ..   url = col_character(),
  ..   author = col_character(),
  ..   date_published = col_date(format = ""),
  ..   ingredients = col_character(),
  ..   calories = col_double(),
  ..   fat = col_double(),
  ..   carbs = col_double(),
  ..   protein = col_double(),
  ..   avg_rating = col_double(),
  ..   total_ratings = col_double(),
  ..   reviews = col_double(),
  ..   prep_time = col_double(),
  ..   cook_time = col_double(),
  ..   total_time = col_double(),
  ..   servings = col_double()
  .. )
 - attr(*, "problems")=<externalptr> 
\end{verbatim}

Isto feito, podemos dizer que o primeiro conjunto, o maior, chamado
\texttt{all\_recipes}, é constituído de 14.426 observações e 16
variáveis; o segundo, \texttt{cuisines}, 2.218 e 17 variáveis. O segundo
conjunto tem, portanto, uma variável a mais, esta sendo a variável
\emph{country}, uma variável de caracteres que descreve o país de origem
da receita. Nas outras variáveis, presentes em ambos os conjuntos de
dados, há informações que descrevem as receitas postadas no site,
trazendo informações dos autores, url's das receitas, ingredientes e
informações nutricionais, tempos de preparo, cozimento e total,
avaliações e vezes reproduzidas. Três das variáveis são do tipo
caracteres, um do tipo data, e o restante valores numéricos.

As observações vêm em uma ordem padrão de índice (definida pelo
fornecedor), mas podem ser facilmente classificadas (reordenadas) de
acordo com as variáveis disponíveis, como, por exemplo, ordem crescente
ou decrescente de calorias, classificação média, ou tempo de preparo,
etc.

\subsubsection{Dicionário de
Variáveis}\label{dicionuxe1rio-de-variuxe1veis}

Abaixo, temos um dicionário das variáveis existentes em ambos os
conjuntos de dados:

\begin{enumerate}
\def\labelenumi{\arabic{enumi}.}
\tightlist
\item
  \texttt{all\_recipes}:
\end{enumerate}

\begin{longtable}[]{@{}
  >{\raggedright\arraybackslash}p{(\linewidth - 4\tabcolsep) * \real{0.2273}}
  >{\raggedright\arraybackslash}p{(\linewidth - 4\tabcolsep) * \real{0.3182}}
  >{\raggedright\arraybackslash}p{(\linewidth - 4\tabcolsep) * \real{0.4545}}@{}}
\toprule\noalign{}
\begin{minipage}[b]{\linewidth}\raggedright
Variável
\end{minipage} & \begin{minipage}[b]{\linewidth}\raggedright
Tipo de Dado
\end{minipage} & \begin{minipage}[b]{\linewidth}\raggedright
Descrição Completa
\end{minipage} \\
\midrule\noalign{}
\endhead
\bottomrule\noalign{}
\endlastfoot
\texttt{author} & Caractere/Fator & Nome do usuário que postou a
receita \\
\texttt{recipe\_name} & Caractere & Nome da receita \\
\texttt{url} & Caractere & Link da receita \\
\texttt{ingredients} & Caractere & Base para contagem de ingredientes \\
\texttt{prep\_time} & Numérico & Tempo de preparo (minutos) \\
\texttt{cook\_time} & Numérico & Tempo de cozimento (minutos) \\
\texttt{total\_time} & Numérico & Tempo total (preparo + cozimento).
Variável chave para Complexidade \\
\texttt{servings} & Numérico & Número de porções que a receita produz \\
\texttt{calories} & Numérico & Conteúdo energético (Kcal). Variável
chave Nutricional \\
\texttt{fat} & Numérico & Quantidade de gordura (g) \\
\texttt{protein} & Numérico & Quantidade de proteína (g) \\
\texttt{carbs} & Numérico & Quantidade de carboidratos (g) \\
\texttt{avg\_rating} & Numérico & Avaliação média (1-5). Métrica de
popularidade \\
\texttt{total\_rating} & Numérico & Número de avaliações. Métrica de
popularidade \\
\texttt{reviews} & Numérico/Inteiro & Número de avaliações recebidas \\
\texttt{date\_published} & Data & Data em que a receita foi publicada no
site \\
\end{longtable}

\begin{enumerate}
\def\labelenumi{\arabic{enumi}.}
\setcounter{enumi}{1}
\tightlist
\item
  \texttt{cuisines}:
\end{enumerate}

\begin{longtable}[]{@{}
  >{\raggedright\arraybackslash}p{(\linewidth - 4\tabcolsep) * \real{0.1800}}
  >{\raggedright\arraybackslash}p{(\linewidth - 4\tabcolsep) * \real{0.5000}}
  >{\raggedright\arraybackslash}p{(\linewidth - 4\tabcolsep) * \real{0.3200}}@{}}
\toprule\noalign{}
\begin{minipage}[b]{\linewidth}\raggedright
Variável
\end{minipage} & \begin{minipage}[b]{\linewidth}\raggedright
Tipo de Dado
\end{minipage} & \begin{minipage}[b]{\linewidth}\raggedright
Descrição Completa
\end{minipage} \\
\midrule\noalign{}
\endhead
\bottomrule\noalign{}
\endlastfoot
\texttt{author} & Caractere/Fator & Nome do usuário \\
\texttt{name} & Caractere & Nome da receita \\
\texttt{url} & Caractere & Link da receita \\
\texttt{ingredients} & Caractere & Usado para derivar a contagem de
ingredientes \\
\texttt{prep\_time} & Numérico & Tempo de preparo (minutos) \\
\texttt{cook\_time} & Numérico & Tempo de cozimento (minutos) \\
\texttt{total\_time} & Numérico & Tempo total (preparo + cozimento) \\
\texttt{servings} & Numérico & Número de porções \\
\texttt{calories} & Numérico & Conteúdo energético (Kcal) \\
\texttt{fat} & Numérico & Quantidade de gordura (g) \\
\texttt{protein} & Numérico & Quantidade de proteína (g) \\
\texttt{carbs} & Numérico & Quantidade de carboidratos (g) \\
\texttt{avg\_rating} & Numérico & Avaliação média (1-5). Métrica de
popularidade \\
\texttt{total\_rating} & Numérico & Número de avaliações. Métrica de
popularidade \\
\texttt{reviews} & Numérico/Inteiro & Número de avaliações recebidas \\
\texttt{date\_published} & Data & Data de publicação \\
\texttt{country} & Caractere/Fator & \textbf{País de origem.} Variável
chave para análise geocultural \\
\end{longtable}

\subsubsection{Tratamento dos Dados}\label{tratamento-dos-dados}

\begin{enumerate}
\def\labelenumi{\arabic{enumi}.}
\item
  Fusão de Conjuntos de Dados: Inicialmente, o dataset principal
  (all\_recipes) foi unido ao dataset auxiliar (cuisines) utilizando-se
  a variável de chave primária (recipe\_id ou equivalente), resultando
  em um único data frame analítico.
\item
  Tratamento de Valores Faltantes (NA): Verificou-se a presença de
  valores NA (Not Available) nas variáveis nutricionais. Foi adotada a
  estratégia de imputação por mediana para os NAs em variáveis
  contínuas, como calorias, onde a ausência de informação era
  considerada Missing At Random (MAR), para maximizar o número de
  observações úteis. Para a variável country, os NAs foram categorizados
  como ``Não Especificado''.
\item
  Padronização e Normalização: Variáveis com alta variabilidade, como
  reproduções, foram consideradas para transformação logarítmica
  \((\log(X+1))\) para aproximar suas distribuições de uma normalidade e
  mitigar a influência de outliers extremos no ajuste dos modelos de
  regressão.
\item
  Codificação de Variáveis Categóricas: A variável country foi
  convertida para o tipo fator (factor) para uso em modelos
  estatísticos, sendo implementado o esquema de codificação dummy
  (variáveis binárias) para as análises de regressão.
\end{enumerate}

\subsubsection{Técnicas Estatísticas e
Computacionais}\label{tuxe9cnicas-estatuxedsticas-e-computacionais}

Detalhando os métodos utilizados para o processamento, análise e
visualização dos dados, podemos começar apontando o fato de que o estudo
foi conduzido utilizando a linguagem R e o ecossistema de pacotes
\texttt{tidyverse} (incluindo \texttt{dplyr} e \texttt{ggplot2}). As
etapas iniciais de preparação de dados incluíram algumas tarefas
distintas. Dentre elas pode-se apontar duas direções principais: a de
padronização geográfica e a de derivação das variáveis. Ambas serão
definidas nos próximos parágrafos.

Primeiramente, a padronização geográfica. Esta consistiu na padronizaçõ
de países e regiões para categorias consistentes (por exemplo, o
agrupamento de sub-culinárias regionais dos EUA sob o rótulo ``US
American''). Posteriormente, as culinárias foram agrupadas em seis
categorias continentais (Ásia, Europa, América do Norte, América do Sul,
África e Oceania) para análises macro.

Em segundo lugar, e menos para a simples categorização de variáveis e
observações, foram criadas duas variáveis-chave para medir a
complexidade e o esforço. A primeira a Quantidade de Ingredientes
(\(\hat{I}\)): Calculada pela contagem de vírgulas nos \emph{strings} da
lista de ingredientes, adicionada de um, como \emph{proxy} da
complexidade. A segunda, o Tempo Total de Preparo (\(\hat{T}\)):
Variável \texttt{total\_time} convertida para formato numérico
(minutos), representando o esforço temporal.

\subsubsection{Métodos Estatísticos e
Modelagem}\label{muxe9todos-estatuxedsticos-e-modelagem}

Olhando para as anáises estatísticas, elas se concentraram em quatro
vertentes principais. A primeira é dedicada à estatística descritiva e
ao ranqueamento de observações. Foram reaizados cálculo de média de
avaliação (\(\bar{R}_{g}\)), desvio padrão (\(s_R\)) e contagem de
receitas (\(N\)) por país/região. Após isto, foi aplicado um filtro
mínimo de \(N \geq 10\) receitas para garantir a estabilidade das médias
de avaliação. A segunda vertente diz respeito à análise das quantidades
de ingredientes e dos tempos de preparo, foi empregada a análise de
correlação, utilizando o Coeficiente de Correlação de Pearson (\(r\))
para quantificar a relação linear entre as variáveis de
complexidade/esforço e a qualidade percebida (média de avaliação).

Além delas, a análise bivariada foi empregada para demonstrar a relação
entre Popularidade e Qualidade. Um Gráfico de Bolhas foi escolhido para
visualizar simultaneamente a média de avaliações por receita
(Popularidade), a média de nota (Qualidade) e o número de receitas
(Tamanho da Bolha) por culinária. Por fim, foi examinada a série
temporal do Total de Avaliações no Ano de Publicação para rastrear a
evolução da popularidade das culinárias ranqueadas ao longo do período
de dados.

\subsubsection{Visualização dos
Dados}\label{visualizauxe7uxe3o-dos-dados}

A visualização de dados é parte central para a interpretação dos
resultados, sendo o pacote \emph{ggplot2} utilizado para a criação de
todos os gráficos estatísticos, como o ranking de culinárias, os
gráficos de dispersão e o Gráfico de Bolhas. Os pacotes \emph{gridExtra}
e \emph{grid} foram cruciais para a renderização e o arranjo das tabelas
comparativas e seus títulos (\emph{tableGrob}, \emph{grid.arrange},
\emph{textGrob}), garantindo a apresentação organizada dos sumários de
dados.

\subsection{Resultados e discussão}\label{sec-03}

\subsubsection{}\label{section}

Os resultados atingidos em cada parte do trabalho seguem em conformidade
com as diferentes direções tomadas em cada análise. Inicialmente, a
análise das médias de avaliação revelou um ranking onde culinárias com
alta qualidade percebida (alta \(\bar{R}_{g}\)) frequentemente exibiam
um baixo desvio padrão (\(s_R\)), sugerindo consistência na excelência.
O agrupamento continental permitiu observar tendências regionais na
satisfação do usuário. A Figura 1 apresenta o ranking das top 15
culinárias, destacando a média de avaliação com barras de erro
representando o desvio padrão. Após análise, A variação na qualidade
entre continentes sugere que a tradição e a identidade gastronômica
específica de uma região exercem maior influência na satisfação do que
fatores universais de preparação.

\begin{Shaded}
\begin{Highlighting}[]
\CommentTok{\# limpeza e organização dos dados}

\NormalTok{cuisines\_limpo }\OtherTok{\textless{}{-}}\NormalTok{ cuisines }\SpecialCharTok{\%\textgreater{}\%}
  \FunctionTok{mutate}\NormalTok{(}
    \AttributeTok{country\_padrao =} \FunctionTok{case\_when}\NormalTok{(}
      \CommentTok{\# Padroniza "Jewish"}
\NormalTok{      country }\SpecialCharTok{\%in\%} \FunctionTok{c}\NormalTok{(}\StringTok{"Jewish"}\NormalTok{) }\SpecialCharTok{\textasciitilde{}} \StringTok{"Israeli/Jewish"}\NormalTok{,}
      \CommentTok{\# Padroniza culinárias regionais dos EUA}
\NormalTok{      country }\SpecialCharTok{\%in\%} \FunctionTok{c}\NormalTok{(}\StringTok{"Cajun and Creole"}\NormalTok{, }\StringTok{"Southern Recipes"}\NormalTok{, }\StringTok{"Tex{-}Mex"}\NormalTok{, }\StringTok{"Southwestern Recipes"}\NormalTok{, }\StringTok{"Amish and Mennonite"}\NormalTok{, }\StringTok{"Soul Food"}\NormalTok{) }\SpecialCharTok{\textasciitilde{}} \StringTok{"US American"}\NormalTok{, }
      \CommentTok{\# Mantém o nome do país para todos os outros}
      \AttributeTok{.default =}\NormalTok{ country }
\NormalTok{    )}
\NormalTok{  )}

\CommentTok{\# Exibe a contagem das categorias padronizadas (opcional)}
\FunctionTok{table}\NormalTok{(cuisines\_limpo}\SpecialCharTok{$}\NormalTok{country\_padrao)}
\end{Highlighting}
\end{Shaded}

\begin{verbatim}

                 Argentinian Australian and New Zealander 
                          30                           65 
                    Austrian                  Bangladeshi 
                          22                           12 
                     Belgian                    Brazilian 
                           6                           67 
                    Canadian                      Chilean 
                          67                           22 
                     Chinese                    Colombian 
                          65                           11 
                       Cuban                       Danish 
                          65                           33 
                       Dutch                     Filipino 
                          22                           66 
                     Finnish                       French 
                          18                           65 
                      German                        Greek 
                          62                           62 
                      Indian                   Indonesian 
                          65                           24 
                     Israeli               Israeli/Jewish 
                          23                           61 
                     Italian                     Jamaican 
                          64                           43 
                    Japanese                       Korean 
                          63                           56 
                    Lebanese                    Malaysian 
                          51                           24 
                   Norwegian                    Pakistani 
                          26                           25 
                     Persian                     Peruvian 
                          45                           38 
                      Polish                   Portuguese 
                          61                           53 
                Puerto Rican                      Russian 
                          60                           65 
                Scandinavian                South African 
                          38                           19 
                     Spanish                      Swedish 
                          61                           31 
                       Swiss                         Thai 
                          10                           62 
                     Turkish                  US American 
                          36                          292 
                  Vietnamese 
                          62 
\end{verbatim}

\begin{Shaded}
\begin{Highlighting}[]
\NormalTok{analise\_por\_pais }\OtherTok{\textless{}{-}}\NormalTok{ cuisines\_limpo }\SpecialCharTok{\%\textgreater{}\%}
  \CommentTok{\# Remove linhas onde a avaliação é NA}
  \FunctionTok{filter}\NormalTok{(}\SpecialCharTok{!}\FunctionTok{is.na}\NormalTok{(avg\_rating)) }\SpecialCharTok{\%\textgreater{}\%}
  \FunctionTok{group\_by}\NormalTok{(country\_padrao) }\SpecialCharTok{\%\textgreater{}\%}
  \FunctionTok{summarise}\NormalTok{(}
    \AttributeTok{contagem\_receitas =} \FunctionTok{n}\NormalTok{(),}
    \AttributeTok{media\_rating =} \FunctionTok{mean}\NormalTok{(avg\_rating, }\AttributeTok{na.rm =} \ConstantTok{TRUE}\NormalTok{),}
    \AttributeTok{desvio\_padrao\_rating =} \FunctionTok{sd}\NormalTok{(avg\_rating, }\AttributeTok{na.rm =} \ConstantTok{TRUE}\NormalTok{), }
    \AttributeTok{media\_porcoes =} \FunctionTok{mean}\NormalTok{(servings, }\AttributeTok{na.rm =} \ConstantTok{TRUE}\NormalTok{),}
    \AttributeTok{.groups =} \StringTok{\textquotesingle{}drop\textquotesingle{}}
\NormalTok{  ) }\SpecialCharTok{\%\textgreater{}\%}
  \CommentTok{\# Filtra países com poucas receitas}
  \FunctionTok{filter}\NormalTok{(contagem\_receitas }\SpecialCharTok{\textgreater{}=} \DecValTok{10}\NormalTok{) }\SpecialCharTok{\%\textgreater{}\%}
  \FunctionTok{arrange}\NormalTok{(}\FunctionTok{desc}\NormalTok{(media\_rating))}

\CommentTok{\# Exibe o resultado da análise}
\FunctionTok{print}\NormalTok{(analise\_por\_pais)}
\end{Highlighting}
\end{Shaded}

\begin{verbatim}
# A tibble: 42 x 5
   country_padrao contagem_receitas media_rating desvio_padrao_rating
   <chr>                      <int>        <dbl>                <dbl>
 1 French                        62         4.69                0.238
 2 Greek                         57         4.68                0.221
 3 Italian                       57         4.65                0.373
 4 Puerto Rican                  59         4.61                0.458
 5 Korean                        55         4.60                0.294
 6 Persian                       38         4.60                0.384
 7 Israeli/Jewish                60         4.59                0.266
 8 Chinese                       64         4.58                0.272
 9 Russian                       65         4.57                0.270
10 US American                  282         4.55                0.354
# i 32 more rows
# i 1 more variable: media_porcoes <dbl>
\end{verbatim}

\begin{Shaded}
\begin{Highlighting}[]
\NormalTok{analise\_por\_continente }\OtherTok{\textless{}{-}}\NormalTok{ analise\_por\_pais }\SpecialCharTok{\%\textgreater{}\%}
  \FunctionTok{mutate}\NormalTok{(}
    \AttributeTok{continente =} \FunctionTok{case\_when}\NormalTok{(}
      \CommentTok{\# === ASIA ===}
\NormalTok{      country\_padrao }\SpecialCharTok{\%in\%} \FunctionTok{c}\NormalTok{(}\StringTok{"Japanese"}\NormalTok{, }\StringTok{"Indian"}\NormalTok{, }\StringTok{"Chinese"}\NormalTok{, }\StringTok{"Thai"}\NormalTok{, }\StringTok{"Filipino"}\NormalTok{, }\StringTok{"Vietnamese"}\NormalTok{, }\StringTok{"Korean"}\NormalTok{, }\StringTok{"Indonesian"}\NormalTok{, }\StringTok{"Turkish"}\NormalTok{, }\StringTok{"Malaysian"}\NormalTok{, }\StringTok{"Pakistani"}\NormalTok{, }\StringTok{"Iranian"}\NormalTok{, }\StringTok{"Israeli/Jewish"}\NormalTok{, }\StringTok{"Lebanese"}\NormalTok{, }\StringTok{"Syrian"}\NormalTok{, }\StringTok{"Yemeni"}\NormalTok{, }\StringTok{"Emirati"}\NormalTok{, }\StringTok{"Kuwaiti"}\NormalTok{, }\StringTok{"Omani"}\NormalTok{, }\StringTok{"Qatari"}\NormalTok{, }\StringTok{"Saudi Arabian"}\NormalTok{, }\StringTok{"Middle Eastern"}\NormalTok{, }\StringTok{"Persian"}\NormalTok{, }\StringTok{"Turkish"}\NormalTok{, }\StringTok{"Israeli"}\NormalTok{, }\StringTok{"Bangladeshi"}\NormalTok{) }\SpecialCharTok{\textasciitilde{}} \StringTok{"Ásia"}\NormalTok{,}
      \CommentTok{\# === EUROPA ===}
\NormalTok{      country\_padrao }\SpecialCharTok{\%in\%} \FunctionTok{c}\NormalTok{(}\StringTok{"British"}\NormalTok{, }\StringTok{"French"}\NormalTok{, }\StringTok{"German"}\NormalTok{, }\StringTok{"Irish"}\NormalTok{, }\StringTok{"Italian"}\NormalTok{, }\StringTok{"Spanish"}\NormalTok{, }\StringTok{"Scandinavian Region"}\NormalTok{, }\StringTok{"Greek"}\NormalTok{, }\StringTok{"Swiss"}\NormalTok{, }\StringTok{"Dutch"}\NormalTok{, }\StringTok{"Austrian"}\NormalTok{, }\StringTok{"Portuguese"}\NormalTok{, }\StringTok{"Russian"}\NormalTok{, }\StringTok{"Belgian"}\NormalTok{, }\StringTok{"Hungarian"}\NormalTok{, }\StringTok{"Polish"}\NormalTok{, }\StringTok{"Czech"}\NormalTok{, }\StringTok{"Slovakian"}\NormalTok{, }\StringTok{"Romanian"}\NormalTok{, }\StringTok{"Ukrainian"}\NormalTok{, }\StringTok{"Scandinavian"}\NormalTok{, }\StringTok{"Finnish"}\NormalTok{, }\StringTok{"Swedish"}\NormalTok{, }\StringTok{"Norwegian"}\NormalTok{, }\StringTok{"Danish"}\NormalTok{) }\SpecialCharTok{\textasciitilde{}} \StringTok{"Europa"}\NormalTok{,}
      \CommentTok{\# === AMÉRICA DO NORTE e CARIBE ===}
\NormalTok{      country\_padrao }\SpecialCharTok{\%in\%} \FunctionTok{c}\NormalTok{(}\StringTok{"US American"}\NormalTok{, }\StringTok{"Canadian"}\NormalTok{, }\StringTok{"Mexican"}\NormalTok{, }\StringTok{"Caribbean Region"}\NormalTok{, }\StringTok{"Puerto Rican"}\NormalTok{, }\StringTok{"Cuban"}\NormalTok{, }\StringTok{"Jamaican"}\NormalTok{, }\StringTok{"Dominican"}\NormalTok{, }\StringTok{"Haitian"}\NormalTok{, }\StringTok{"Costa Rican"}\NormalTok{, }\StringTok{"Panamanian"}\NormalTok{, }\StringTok{"Honduran"}\NormalTok{, }\StringTok{"Guatemalan"}\NormalTok{, }\StringTok{"Salvadoran"}\NormalTok{) }\SpecialCharTok{\textasciitilde{}} \StringTok{"América do Norte"}\NormalTok{,}
      \CommentTok{\# === AMÉRICA DO SUL ===}
\NormalTok{      country\_padrao }\SpecialCharTok{\%in\%} \FunctionTok{c}\NormalTok{(}\StringTok{"Chilean"}\NormalTok{, }\StringTok{"Brazilian"}\NormalTok{, }\StringTok{"Peruvian"}\NormalTok{, }\StringTok{"Argentinian"}\NormalTok{, }\StringTok{"Venezuelan"}\NormalTok{, }\StringTok{"Colombian"}\NormalTok{, }\StringTok{"Ecuadorian"}\NormalTok{, }\StringTok{"Bolivian"}\NormalTok{, }\StringTok{"Paraguayan"}\NormalTok{, }\StringTok{"Uruguayan"}\NormalTok{) }\SpecialCharTok{\textasciitilde{}} \StringTok{"América do Sul"}\NormalTok{,}
      \CommentTok{\# === ÁFRICA ===}
\NormalTok{      country\_padrao }\SpecialCharTok{\%in\%} \FunctionTok{c}\NormalTok{(}\StringTok{"Moroccan"}\NormalTok{, }\StringTok{"Egyptian"}\NormalTok{, }\StringTok{"South African"}\NormalTok{, }\StringTok{"Ethiopian"}\NormalTok{, }\StringTok{"Kenyan"}\NormalTok{, }\StringTok{"Nigerian"}\NormalTok{, }\StringTok{"Algerian"}\NormalTok{, }\StringTok{"Tunisian"}\NormalTok{, }\StringTok{"Somali"}\NormalTok{, }\StringTok{"African"}\NormalTok{) }\SpecialCharTok{\textasciitilde{}} \StringTok{"África"}\NormalTok{,}
      \CommentTok{\# === OCEANIA ===}
\NormalTok{      country\_padrao }\SpecialCharTok{\%in\%} \FunctionTok{c}\NormalTok{(}\StringTok{"Australian and New Zealander"}\NormalTok{, }\StringTok{"Fijian"}\NormalTok{, }\StringTok{"Samoan"}\NormalTok{) }\SpecialCharTok{\textasciitilde{}} \StringTok{"Oceania"}\NormalTok{,}
      \AttributeTok{.default =} \StringTok{"Outros"} 
\NormalTok{    )}
\NormalTok{  )}

\CommentTok{\# Agrupamento FINAL}
\NormalTok{analise\_por\_continente\_agregado }\OtherTok{\textless{}{-}}\NormalTok{ analise\_por\_continente }\SpecialCharTok{\%\textgreater{}\%}
  \FunctionTok{group\_by}\NormalTok{(continente) }\SpecialCharTok{\%\textgreater{}\%}
  \FunctionTok{summarise}\NormalTok{(}
    \AttributeTok{contagem\_total =} \FunctionTok{sum}\NormalTok{(contagem\_receitas),}
    \AttributeTok{media\_rating\_continente =} \FunctionTok{mean}\NormalTok{(media\_rating, }\AttributeTok{na.rm =} \ConstantTok{TRUE}\NormalTok{),}
    \AttributeTok{desvio\_padrao\_continente =} \FunctionTok{mean}\NormalTok{(desvio\_padrao\_rating, }\AttributeTok{na.rm =} \ConstantTok{TRUE}\NormalTok{), }
    \AttributeTok{.groups =} \StringTok{\textquotesingle{}drop\textquotesingle{}}
\NormalTok{  ) }\SpecialCharTok{\%\textgreater{}\%}
  \FunctionTok{arrange}\NormalTok{(}\FunctionTok{desc}\NormalTok{(media\_rating\_continente))}

\CommentTok{\# Visualizando a Tabela Agregada}
\FunctionTok{print}\NormalTok{(analise\_por\_continente\_agregado)}
\end{Highlighting}
\end{Shaded}

\begin{verbatim}
# A tibble: 6 x 4
  continente       contagem_total media_rating_continente desvio_padrao_contin~1
  <chr>                     <int>                   <dbl>                  <dbl>
1 América do Norte            515                    4.54                  0.362
2 Europa                      658                    4.52                  0.356
3 África                       19                    4.49                  0.371
4 Ásia                        697                    4.49                  0.383
5 Oceania                      62                    4.36                  0.478
6 América do Sul              146                    4.34                  0.515
# i abbreviated name: 1: desvio_padrao_continente
\end{verbatim}

\begin{Shaded}
\begin{Highlighting}[]
\NormalTok{grafico\_colorido }\OtherTok{\textless{}{-}}\NormalTok{ analise\_por\_continente }\SpecialCharTok{\%\textgreater{}\%}
  \CommentTok{\# Limita aos Top 15 países/regiões para visualização}
  \FunctionTok{slice\_max}\NormalTok{(media\_rating, }\AttributeTok{n =} \DecValTok{15}\NormalTok{) }\SpecialCharTok{\%\textgreater{}\%} 
  
  \CommentTok{\# Estética principal, usando \textquotesingle{}fill\textquotesingle{} para o continente}
  \FunctionTok{ggplot}\NormalTok{(}\FunctionTok{aes}\NormalTok{(}\AttributeTok{x =} \FunctionTok{reorder}\NormalTok{(country\_padrao, media\_rating), }
             \AttributeTok{y =}\NormalTok{ media\_rating, }
             \AttributeTok{fill =}\NormalTok{ continente)) }\SpecialCharTok{+}
  
  \CommentTok{\# Adiciona as barras coloridas}
  \FunctionTok{geom\_col}\NormalTok{(}\AttributeTok{alpha =} \FloatTok{0.8}\NormalTok{) }\SpecialCharTok{+}
  
  \CommentTok{\# Adiciona as Barras de Erro (Desvio Padrão)}
  \FunctionTok{geom\_errorbar}\NormalTok{(}\FunctionTok{aes}\NormalTok{(}\AttributeTok{ymin =}\NormalTok{ media\_rating }\SpecialCharTok{{-}}\NormalTok{ desvio\_padrao\_rating, }
                    \AttributeTok{ymax =}\NormalTok{ media\_rating }\SpecialCharTok{+}\NormalTok{ desvio\_padrao\_rating),}
                \AttributeTok{width =} \FloatTok{0.2}\NormalTok{, }\AttributeTok{color =} \StringTok{"black"}\NormalTok{) }\SpecialCharTok{+} \CommentTok{\# Cor das barras de erro}
  
  \CommentTok{\# Escala de cores (Set2 é uma paleta amigável)}
  \FunctionTok{scale\_fill\_brewer}\NormalTok{(}\AttributeTok{palette =} \StringTok{"Set2"}\NormalTok{) }\SpecialCharTok{+}
  
  \CommentTok{\# Inverte os eixos para facilitar a leitura}
  \FunctionTok{coord\_flip}\NormalTok{() }\SpecialCharTok{+}
  
  \CommentTok{\# Títulos e Rótulos}
  \FunctionTok{labs}\NormalTok{(}
    \AttributeTok{title =} \StringTok{"Top 15 Culinárias por Média de Avaliação (Coloridas por Continente)"}\NormalTok{,}
    \AttributeTok{subtitle =} \StringTok{"Barras de erro representam o Desvio Padrão do Rating."}\NormalTok{,}
    \AttributeTok{x =} \StringTok{"País/Região"}\NormalTok{,}
    \AttributeTok{y =} \StringTok{"Média de Avaliação (1 a 5)"}\NormalTok{,}
    \AttributeTok{fill =} \StringTok{"Continente"} \CommentTok{\# Legenda da cor}
\NormalTok{  ) }\SpecialCharTok{+}
  
  \CommentTok{\# Tema}
  \FunctionTok{theme\_minimal}\NormalTok{() }\SpecialCharTok{+}
  \FunctionTok{theme}\NormalTok{(}\AttributeTok{plot.title =} \FunctionTok{element\_text}\NormalTok{(}\AttributeTok{face =} \StringTok{"bold"}\NormalTok{))}

\FunctionTok{print}\NormalTok{(grafico\_colorido)}
\end{Highlighting}
\end{Shaded}

\pandocbounded{\includegraphics[keepaspectratio]{Trabalho-Final---Elementos-de-Programação_files/figure-pdf/unnamed-chunk-5-1.pdf}}

\subsubsection{Relação entre Ingredientes e
Avaliações}\label{relauxe7uxe3o-entre-ingredientes-e-avaliauxe7uxf5es}

A análise dos dados revela uma correlação positiva, porém extremamente
fraca, entre a quantidade de ingredientes e a avaliação final das
receitas em ambos os bancos de dados. A Figura 2 confirma que, com
coeficientes de 0,1 para ``cuisines'' e 0,08 para ``all\_recipes'',
conclui-se que a complexidade do prato (medida pelo número de itens)
praticamente não influencia a nota atribuída pelos usuários; ou seja,
embora exista uma tendência matemática muito sutil de aumento na
avaliação conforme o número de ingredientes cresce, na prática essa
relação é desprezível, indicando que receitas simples têm tanta chance
de serem bem avaliadas quanto as mais complexas.

\begin{Shaded}
\begin{Highlighting}[]
\CommentTok{\#ingredientes x nota}
\NormalTok{ingredientes\_nota }\OtherTok{=} \ControlFlowTok{function}\NormalTok{(dados, nome\_do\_banco) \{}
  
  \CommentTok{\#coluna de quantidade de ingredientes}
\NormalTok{  dados\_processados }\OtherTok{=}\NormalTok{ dados }\SpecialCharTok{\%\textgreater{}\%}
    \FunctionTok{filter}\NormalTok{(}\SpecialCharTok{!}\FunctionTok{is.na}\NormalTok{(avg\_rating), }\SpecialCharTok{!}\FunctionTok{is.na}\NormalTok{(ingredients)) }\SpecialCharTok{\%\textgreater{}\%}
    \FunctionTok{mutate}\NormalTok{(}
      \AttributeTok{qtd\_ingredientes =} \FunctionTok{str\_count}\NormalTok{(ingredients, }\StringTok{","}\NormalTok{) }\SpecialCharTok{+} \DecValTok{1}\NormalTok{,}
      \AttributeTok{avg\_rating =} \FunctionTok{as.numeric}\NormalTok{(avg\_rating)}
\NormalTok{    )}
  
  \CommentTok{\# Correlação}
\NormalTok{  correlacao }\OtherTok{=} \FunctionTok{cor}\NormalTok{(dados\_processados}\SpecialCharTok{$}\NormalTok{qtd\_ingredientes, }
\NormalTok{                   dados\_processados}\SpecialCharTok{$}\NormalTok{avg\_rating, }
                   \AttributeTok{use =} \StringTok{"complete.obs"}\NormalTok{)}
  
  \FunctionTok{cat}\NormalTok{(}\FunctionTok{paste0}\NormalTok{(}\StringTok{"}\SpecialCharTok{\textbackslash{}n}\StringTok{Análise para: "}\NormalTok{, nome\_do\_banco, }\StringTok{"}\SpecialCharTok{\textbackslash{}n}\StringTok{"}\NormalTok{))}
  \FunctionTok{cat}\NormalTok{(}\FunctionTok{paste}\NormalTok{(}\StringTok{"Correlação:"}\NormalTok{, }\FunctionTok{round}\NormalTok{(correlacao, }\DecValTok{3}\NormalTok{), }\StringTok{"}\SpecialCharTok{\textbackslash{}n}\StringTok{"}\NormalTok{))}
  
  \CommentTok{\# Gráfico}
\NormalTok{  grafico }\OtherTok{=} \FunctionTok{ggplot}\NormalTok{(dados\_processados, }\FunctionTok{aes}\NormalTok{(}\AttributeTok{x =}\NormalTok{ qtd\_ingredientes, }\AttributeTok{y =}\NormalTok{ avg\_rating)) }\SpecialCharTok{+}
    \FunctionTok{geom\_jitter}\NormalTok{(}\AttributeTok{alpha =} \FloatTok{0.2}\NormalTok{, }\AttributeTok{color =} \StringTok{"steelblue"}\NormalTok{) }\SpecialCharTok{+} 
    \FunctionTok{geom\_smooth}\NormalTok{(}\AttributeTok{method =} \StringTok{"lm"}\NormalTok{, }\AttributeTok{color =} \StringTok{"darkred"}\NormalTok{) }\SpecialCharTok{+} 
    \FunctionTok{labs}\NormalTok{(}
      \AttributeTok{title =} \FunctionTok{paste}\NormalTok{(}\StringTok{"Ingredientes vs Nota:"}\NormalTok{, nome\_do\_banco),}
      \AttributeTok{subtitle =} \FunctionTok{paste}\NormalTok{(}\StringTok{"Correlação de Pearson:"}\NormalTok{, }\FunctionTok{round}\NormalTok{(correlacao, }\DecValTok{3}\NormalTok{)),}
      \AttributeTok{x =} \StringTok{"Quantidade de Ingredientes"}\NormalTok{,}
      \AttributeTok{y =} \StringTok{"Nota Média"}
\NormalTok{    ) }\SpecialCharTok{+}
    \FunctionTok{theme\_minimal}\NormalTok{()}
  
  \FunctionTok{print}\NormalTok{(grafico)}
\NormalTok{\}}

\FunctionTok{ingredientes\_nota}\NormalTok{(cuisines, }\StringTok{"Banco Cuisines"}\NormalTok{)}
\end{Highlighting}
\end{Shaded}

\begin{verbatim}

Análise para: Banco Cuisines
Correlação: 0.1 
\end{verbatim}

\begin{verbatim}
`geom_smooth()` using formula = 'y ~ x'
\end{verbatim}

\pandocbounded{\includegraphics[keepaspectratio]{Trabalho-Final---Elementos-de-Programação_files/figure-pdf/unnamed-chunk-6-1.pdf}}

\begin{Shaded}
\begin{Highlighting}[]
\FunctionTok{ingredientes\_nota}\NormalTok{(all\_recipes, }\StringTok{"Banco All Recipes"}\NormalTok{)}
\end{Highlighting}
\end{Shaded}

\begin{verbatim}

Análise para: Banco All Recipes
Correlação: 0.08 
\end{verbatim}

\begin{verbatim}
`geom_smooth()` using formula = 'y ~ x'
\end{verbatim}

\pandocbounded{\includegraphics[keepaspectratio]{Trabalho-Final---Elementos-de-Programação_files/figure-pdf/unnamed-chunk-6-2.pdf}}

\subsubsection{Relação entre Tempo de Preparo e
Avaliações}\label{relauxe7uxe3o-entre-tempo-de-preparo-e-avaliauxe7uxf5es}

Ao analisar a influência do tempo total de preparo (total\_time) sobre a
avaliação das receitas (avg\_rating), com uma visualização (Figura 3)
abrangendo até 480 minutos (8 horas), observou-se uma ausência de
correlação significativa em ambos os bancos de dados. Com coeficientes
irrelevantes de 0,047 para o banco `cuisines' e 0,008 para
`all\_recipes', os dados demonstram estatisticamente que a duração da
receita não é um fator determinante para a sua nota final. Isso sugere
que a satisfação dos usuários independe do tempo investido na cozinha,
com pratos rápidos recebendo avaliações tão altas quanto preparos de
longa duração.

\begin{Shaded}
\begin{Highlighting}[]
\NormalTok{tempo\_nota }\OtherTok{\textless{}{-}} \ControlFlowTok{function}\NormalTok{(dados, nome\_do\_banco) \{}
  
  \CommentTok{\# 1. Limpeza e Tratamento}
\NormalTok{  dados\_processados }\OtherTok{\textless{}{-}}\NormalTok{ dados }\SpecialCharTok{\%\textgreater{}\%}
    \FunctionTok{mutate}\NormalTok{(}
      \AttributeTok{avg\_rating =} \FunctionTok{as.numeric}\NormalTok{(avg\_rating),}
      \AttributeTok{total\_time =} \FunctionTok{as.numeric}\NormalTok{(total\_time)}
\NormalTok{    ) }\SpecialCharTok{\%\textgreater{}\%}
    \CommentTok{\# Removemos linhas vazias ou com tempo zerado/negativo}
    \FunctionTok{filter}\NormalTok{(}\SpecialCharTok{!}\FunctionTok{is.na}\NormalTok{(avg\_rating), }\SpecialCharTok{!}\FunctionTok{is.na}\NormalTok{(total\_time), total\_time }\SpecialCharTok{\textgreater{}} \DecValTok{0}\NormalTok{)}
  
  \CommentTok{\# 2. Cálculo da Correlação (considerando todos os dados)}
\NormalTok{  correlacao }\OtherTok{\textless{}{-}} \FunctionTok{cor}\NormalTok{(dados\_processados}\SpecialCharTok{$}\NormalTok{total\_time, }
\NormalTok{                    dados\_processados}\SpecialCharTok{$}\NormalTok{avg\_rating, }
                    \AttributeTok{use =} \StringTok{"complete.obs"}\NormalTok{)}
  
  \FunctionTok{cat}\NormalTok{(}\FunctionTok{paste0}\NormalTok{(}\StringTok{"}\SpecialCharTok{\textbackslash{}n}\StringTok{{-}{-}{-} Análise: "}\NormalTok{, nome\_do\_banco, }\StringTok{" {-}{-}{-}}\SpecialCharTok{\textbackslash{}n}\StringTok{"}\NormalTok{))}
  \FunctionTok{cat}\NormalTok{(}\FunctionTok{paste}\NormalTok{(}\StringTok{"Correlação geral:"}\NormalTok{, }\FunctionTok{round}\NormalTok{(correlacao, }\DecValTok{3}\NormalTok{), }\StringTok{"}\SpecialCharTok{\textbackslash{}n}\StringTok{"}\NormalTok{))}
  
  \CommentTok{\# 3. Gráfico de Dispersão}
  \CommentTok{\# OBS: Fiz um filtro NO GRÁFICO para mostrar apenas receitas até 480 min (8h)}
  \CommentTok{\# Se não fizer isso, uma única receita de 24h esmaga o gráfico todo.}
\NormalTok{  grafico }\OtherTok{\textless{}{-}} \FunctionTok{ggplot}\NormalTok{(dados\_processados }\SpecialCharTok{\%\textgreater{}\%} \FunctionTok{filter}\NormalTok{(total\_time }\SpecialCharTok{\textless{}=} \DecValTok{480}\NormalTok{), }
                    \FunctionTok{aes}\NormalTok{(}\AttributeTok{x =}\NormalTok{ total\_time, }\AttributeTok{y =}\NormalTok{ avg\_rating)) }\SpecialCharTok{+}
    
    \FunctionTok{geom\_jitter}\NormalTok{(}\AttributeTok{alpha =} \FloatTok{0.2}\NormalTok{, }\AttributeTok{color =} \StringTok{"forestgreen"}\NormalTok{) }\SpecialCharTok{+} \CommentTok{\# Pontos verdes}
    \FunctionTok{geom\_smooth}\NormalTok{(}\AttributeTok{method =} \StringTok{"lm"}\NormalTok{, }\AttributeTok{color =} \StringTok{"black"}\NormalTok{) }\SpecialCharTok{+}    \CommentTok{\# Linha de tendência preta}
    
    \FunctionTok{labs}\NormalTok{(}
      \AttributeTok{title =} \FunctionTok{paste}\NormalTok{(}\StringTok{"Tempo de Preparo vs Nota:"}\NormalTok{, nome\_do\_banco),}
      \AttributeTok{subtitle =} \FunctionTok{paste}\NormalTok{(}\StringTok{"Correlação de Pearson:"}\NormalTok{, }\FunctionTok{round}\NormalTok{(correlacao, }\DecValTok{3}\NormalTok{)),}
      \AttributeTok{x =} \StringTok{"Tempo de Preparo (minutos)"}\NormalTok{,}
      \AttributeTok{y =} \StringTok{"Nota (0{-}5)"}
\NormalTok{    ) }\SpecialCharTok{+}
    \FunctionTok{theme\_minimal}\NormalTok{()}
  
  \FunctionTok{print}\NormalTok{(grafico)}
\NormalTok{\}}

\CommentTok{\# Executando para os dois bancos}
\FunctionTok{tempo\_nota}\NormalTok{(cuisines, }\StringTok{"Banco Cuisines"}\NormalTok{)}
\end{Highlighting}
\end{Shaded}

\begin{verbatim}

--- Análise: Banco Cuisines ---
Correlação geral: 0.047 
\end{verbatim}

\begin{verbatim}
`geom_smooth()` using formula = 'y ~ x'
\end{verbatim}

\pandocbounded{\includegraphics[keepaspectratio]{Trabalho-Final---Elementos-de-Programação_files/figure-pdf/unnamed-chunk-7-1.pdf}}

\begin{Shaded}
\begin{Highlighting}[]
\FunctionTok{tempo\_nota}\NormalTok{(all\_recipes, }\StringTok{"Banco All Recipes"}\NormalTok{)}
\end{Highlighting}
\end{Shaded}

\begin{verbatim}

--- Análise: Banco All Recipes ---
Correlação geral: 0.008 
\end{verbatim}

\begin{verbatim}
`geom_smooth()` using formula = 'y ~ x'
\end{verbatim}

\pandocbounded{\includegraphics[keepaspectratio]{Trabalho-Final---Elementos-de-Programação_files/figure-pdf/unnamed-chunk-7-2.pdf}}

\subsubsection{Relação entre Culinárias e Qualidade de acordo com uma
análise
bivarida}\label{relauxe7uxe3o-entre-culinuxe1rias-e-qualidade-de-acordo-com-uma-anuxe1lise-bivarida}

Ao olharmos para a análise bivariada, por fim, estabeleceu-se uma clara
diferenciação entre as culinárias mais populares e as de maior
qualidade. O Gráfico de Bolhas (Figura 4) ilustra que a alta
Popularidade, medida pela média de avaliações por receita, nem sempre se
traduz em alta Qualidade, medida pela média de nota. Esta divergência
indica que a popularidade pode ser impulsionada por fatores de
acessibilidade ou tendência, desvinculando-se da excelência na
avaliação. A análise de tendência temporal complementou este achado ao
mostrar que o Total de Avaliações no Ano é dinâmico, refletindo o ciclo
de vida e a ascensão do interesse em certas culinárias ao longo do
tempo.

\begin{Shaded}
\begin{Highlighting}[]
\CommentTok{\# 1) Instalar e carregar pacotes}
\NormalTok{packages }\OtherTok{\textless{}{-}} \FunctionTok{c}\NormalTok{(}\StringTok{"dplyr"}\NormalTok{, }\StringTok{"ggplot2"}\NormalTok{, }\StringTok{"readr"}\NormalTok{, }\StringTok{"DT"}\NormalTok{)}

\CommentTok{\# Verifica e instala pacotes faltantes (opcional em R Markdown)}
\NormalTok{installed }\OtherTok{\textless{}{-}}\NormalTok{ packages }\SpecialCharTok{\%in\%} \FunctionTok{rownames}\NormalTok{(}\FunctionTok{installed.packages}\NormalTok{())}
\ControlFlowTok{if}\NormalTok{ (}\FunctionTok{any}\NormalTok{(}\SpecialCharTok{!}\NormalTok{installed)) \{}
  \FunctionTok{install.packages}\NormalTok{(packages[}\SpecialCharTok{!}\NormalTok{installed])}
\NormalTok{\}}

\CommentTok{\# Carrega os pacotes}
\FunctionTok{lapply}\NormalTok{(packages, library, }\AttributeTok{character.only =} \ConstantTok{TRUE}\NormalTok{)}
\end{Highlighting}
\end{Shaded}

\begin{verbatim}
Warning: pacote 'DT' foi compilado no R versão 4.5.2
\end{verbatim}

\begin{verbatim}
[[1]]
 [1] "tidytuesdayR" "lubridate"    "forcats"      "stringr"      "dplyr"       
 [6] "purrr"        "readr"        "tidyr"        "tibble"       "ggplot2"     
[11] "tidyverse"    "stats"        "graphics"     "grDevices"    "utils"       
[16] "datasets"     "methods"      "base"        

[[2]]
 [1] "tidytuesdayR" "lubridate"    "forcats"      "stringr"      "dplyr"       
 [6] "purrr"        "readr"        "tidyr"        "tibble"       "ggplot2"     
[11] "tidyverse"    "stats"        "graphics"     "grDevices"    "utils"       
[16] "datasets"     "methods"      "base"        

[[3]]
 [1] "tidytuesdayR" "lubridate"    "forcats"      "stringr"      "dplyr"       
 [6] "purrr"        "readr"        "tidyr"        "tibble"       "ggplot2"     
[11] "tidyverse"    "stats"        "graphics"     "grDevices"    "utils"       
[16] "datasets"     "methods"      "base"        

[[4]]
 [1] "DT"           "tidytuesdayR" "lubridate"    "forcats"      "stringr"     
 [6] "dplyr"        "purrr"        "readr"        "tidyr"        "tibble"      
[11] "ggplot2"      "tidyverse"    "stats"        "graphics"     "grDevices"   
[16] "utils"        "datasets"     "methods"      "base"        
\end{verbatim}

\begin{Shaded}
\begin{Highlighting}[]
\CommentTok{\# 2) Carregar e preparar os dados}
\NormalTok{cuisines }\OtherTok{\textless{}{-}}\NormalTok{ readr}\SpecialCharTok{::}\FunctionTok{read\_csv}\NormalTok{(}
  \StringTok{"https://raw.githubusercontent.com/rfordatascience/tidytuesday/main/data/2025/2025{-}09{-}16/cuisines.csv"}
\NormalTok{)}
\end{Highlighting}
\end{Shaded}

\begin{verbatim}
Rows: 2218 Columns: 17
-- Column specification --------------------------------------------------------
Delimiter: ","
chr   (5): name, country, url, author, ingredients
dbl  (11): calories, fat, carbs, protein, avg_rating, total_ratings, reviews...
date  (1): date_published

i Use `spec()` to retrieve the full column specification for this data.
i Specify the column types or set `show_col_types = FALSE` to quiet this message.
\end{verbatim}

\begin{Shaded}
\begin{Highlighting}[]
\CommentTok{\# Ajustar data e extrair ano}
\NormalTok{cuisines }\OtherTok{\textless{}{-}}\NormalTok{ cuisines }\SpecialCharTok{|\textgreater{}}
  \FunctionTok{mutate}\NormalTok{(}
    \AttributeTok{date\_published =} \FunctionTok{as.Date}\NormalTok{(date\_published),}
    \AttributeTok{year =} \FunctionTok{as.integer}\NormalTok{(}\FunctionTok{format}\NormalTok{(date\_published, }\StringTok{"\%Y"}\NormalTok{))}
\NormalTok{  )}

\NormalTok{year\_min }\OtherTok{\textless{}{-}} \FunctionTok{min}\NormalTok{(cuisines}\SpecialCharTok{$}\NormalTok{year, }\AttributeTok{na.rm =} \ConstantTok{TRUE}\NormalTok{)}
\NormalTok{year\_max }\OtherTok{\textless{}{-}} \FunctionTok{max}\NormalTok{(cuisines}\SpecialCharTok{$}\NormalTok{year, }\AttributeTok{na.rm =} \ConstantTok{TRUE}\NormalTok{)}

\CommentTok{\# Resumo por país/cozinha (usado para ranking e popularidade x qualidade)}
\NormalTok{cuisine\_country\_summary }\OtherTok{\textless{}{-}}\NormalTok{ cuisines }\SpecialCharTok{|\textgreater{}}
  \FunctionTok{group\_by}\NormalTok{(country) }\SpecialCharTok{|\textgreater{}}
  \FunctionTok{summarise}\NormalTok{(}
    \AttributeTok{n\_receitas      =} \FunctionTok{n}\NormalTok{(),}
    \AttributeTok{media\_rating    =} \FunctionTok{mean}\NormalTok{(avg\_rating, }\AttributeTok{na.rm =} \ConstantTok{TRUE}\NormalTok{),}
    \AttributeTok{total\_avaliacoes =} \FunctionTok{sum}\NormalTok{(total\_ratings, }\AttributeTok{na.rm =} \ConstantTok{TRUE}\NormalTok{),}
    \AttributeTok{media\_avaliacoes =} \FunctionTok{mean}\NormalTok{(total\_ratings, }\AttributeTok{na.rm =} \ConstantTok{TRUE}\NormalTok{),}
    \AttributeTok{total\_reviews   =} \FunctionTok{sum}\NormalTok{(reviews, }\AttributeTok{na.rm =} \ConstantTok{TRUE}\NormalTok{),}
    \AttributeTok{media\_reviews   =} \FunctionTok{mean}\NormalTok{(reviews, }\AttributeTok{na.rm =} \ConstantTok{TRUE}\NormalTok{),}
    \AttributeTok{.groups =} \StringTok{"drop"}
\NormalTok{  )}

\CommentTok{\# Dados por país e ano (para tendência)}
\NormalTok{cuisine\_year }\OtherTok{\textless{}{-}}\NormalTok{ cuisines }\SpecialCharTok{|\textgreater{}}
  \FunctionTok{filter}\NormalTok{(}\SpecialCharTok{!}\FunctionTok{is.na}\NormalTok{(year)) }\SpecialCharTok{|\textgreater{}}
  \FunctionTok{group\_by}\NormalTok{(country, year) }\SpecialCharTok{|\textgreater{}}
  \FunctionTok{summarise}\NormalTok{(}
    \AttributeTok{n\_receitas       =} \FunctionTok{n}\NormalTok{(),}
    \AttributeTok{total\_avaliacoes =} \FunctionTok{sum}\NormalTok{(total\_ratings, }\AttributeTok{na.rm =} \ConstantTok{TRUE}\NormalTok{),}
    \AttributeTok{.groups =} \StringTok{"drop"}
\NormalTok{  )}

\CommentTok{\# {-}{-}{-} Definições para o R Markdown (Valores de Filtro Fixo) {-}{-}{-}}
\CommentTok{\# Como não temos os inputs do Shiny, definimos valores fixos para simular os filtros:}
\NormalTok{MIN\_RECEITAS\_FIXO }\OtherTok{\textless{}{-}} \DecValTok{10}  \CommentTok{\# Simula input$min\_receitas}
\NormalTok{METRICA\_RANK\_FIXO }\OtherTok{\textless{}{-}} \StringTok{"total\_avaliacoes"} \CommentTok{\# Simula input$metrica\_rank}
\NormalTok{TOP\_N\_FIXO        }\OtherTok{\textless{}{-}} \DecValTok{10}   \CommentTok{\# Simula input$top\_n}
\NormalTok{ANO\_MIN\_FIXO      }\OtherTok{\textless{}{-}}\NormalTok{ year\_min }\CommentTok{\# Simula input$ano\_range[1]}
\NormalTok{ANO\_MAX\_FIXO      }\OtherTok{\textless{}{-}}\NormalTok{ year\_max }\CommentTok{\# Simula input$ano\_range[2]}
\NormalTok{PAIS\_TENDENCIA\_FIXO }\OtherTok{\textless{}{-}} \StringTok{"top"} \CommentTok{\# Simula input$pais\_tendencia}

\CommentTok{\# Aplica filtro fixo de mínimo de receitas}
\NormalTok{resumo\_filtrado }\OtherTok{\textless{}{-}}\NormalTok{ cuisine\_country\_summary }\SpecialCharTok{|\textgreater{}}
  \FunctionTok{filter}\NormalTok{(n\_receitas }\SpecialCharTok{\textgreater{}=}\NormalTok{ MIN\_RECEITAS\_FIXO)}

\CommentTok{\# Cria o ranking fixo}
\NormalTok{col\_ord }\OtherTok{\textless{}{-}} \ControlFlowTok{switch}\NormalTok{(}
\NormalTok{  METRICA\_RANK\_FIXO,}
  \StringTok{"total\_avaliacoes"} \OtherTok{=}\NormalTok{ resumo\_filtrado}\SpecialCharTok{$}\NormalTok{total\_avaliacoes,}
  \StringTok{"n\_receitas"}       \OtherTok{=}\NormalTok{ resumo\_filtrado}\SpecialCharTok{$}\NormalTok{n\_receitas,}
  \StringTok{"media\_rating"}     \OtherTok{=}\NormalTok{ resumo\_filtrado}\SpecialCharTok{$}\NormalTok{media\_rating}
\NormalTok{)}

\NormalTok{ranking\_paises }\OtherTok{\textless{}{-}}\NormalTok{ resumo\_filtrado }\SpecialCharTok{|\textgreater{}}
  \FunctionTok{arrange}\NormalTok{(}\FunctionTok{desc}\NormalTok{(col\_ord)) }\SpecialCharTok{|\textgreater{}}
  \FunctionTok{slice\_head}\NormalTok{(}\AttributeTok{n =}\NormalTok{ TOP\_N\_FIXO)}
\end{Highlighting}
\end{Shaded}

\begin{Shaded}
\begin{Highlighting}[]
\CommentTok{\# Gráfico de ranking por país}
\NormalTok{df\_ranking }\OtherTok{\textless{}{-}}\NormalTok{ ranking\_paises}

\NormalTok{y\_lab }\OtherTok{\textless{}{-}} \ControlFlowTok{switch}\NormalTok{(}
\NormalTok{  METRICA\_RANK\_FIXO,}
  \StringTok{"total\_avaliacoes"} \OtherTok{=} \StringTok{"Total de avaliações"}\NormalTok{,}
  \StringTok{"n\_receitas"}       \OtherTok{=} \StringTok{"Número de receitas"}\NormalTok{,}
  \StringTok{"media\_rating"}     \OtherTok{=} \StringTok{"Média de nota"}
\NormalTok{)}

\FunctionTok{ggplot}\NormalTok{(df\_ranking, }\FunctionTok{aes}\NormalTok{(}\AttributeTok{x =} \FunctionTok{reorder}\NormalTok{(country, }
                                   \ControlFlowTok{if}\NormalTok{ (METRICA\_RANK\_FIXO }\SpecialCharTok{==} \StringTok{"media\_rating"}\NormalTok{) media\_rating }\ControlFlowTok{else} \FunctionTok{get}\NormalTok{(METRICA\_RANK\_FIXO)),}
                       \AttributeTok{y =} \ControlFlowTok{if}\NormalTok{ (METRICA\_RANK\_FIXO }\SpecialCharTok{==} \StringTok{"media\_rating"}\NormalTok{) media\_rating }\ControlFlowTok{else} \FunctionTok{get}\NormalTok{(METRICA\_RANK\_FIXO))) }\SpecialCharTok{+}
  \FunctionTok{geom\_col}\NormalTok{(}\AttributeTok{fill =} \StringTok{"\#56B4E9"}\NormalTok{) }\SpecialCharTok{+}
  \FunctionTok{coord\_flip}\NormalTok{() }\SpecialCharTok{+}
  \FunctionTok{labs}\NormalTok{(}
    \AttributeTok{title =} \FunctionTok{paste}\NormalTok{(}\StringTok{"Top"}\NormalTok{, TOP\_N\_FIXO, }\StringTok{"países / cozinhas por"}\NormalTok{, y\_lab),}
    \AttributeTok{x =} \StringTok{"País / cozinha"}\NormalTok{,}
    \AttributeTok{y =}\NormalTok{ y\_lab}
\NormalTok{  ) }\SpecialCharTok{+}
  \FunctionTok{theme\_minimal}\NormalTok{()}
\end{Highlighting}
\end{Shaded}

\pandocbounded{\includegraphics[keepaspectratio]{Trabalho-Final---Elementos-de-Programação_files/figure-pdf/unnamed-chunk-10-1.pdf}}

\begin{Shaded}
\begin{Highlighting}[]
\CommentTok{\# Gráfico: popularidade x qualidade por culinária}
\NormalTok{df\_pop\_rating }\OtherTok{\textless{}{-}}\NormalTok{ resumo\_filtrado}

\FunctionTok{ggplot}\NormalTok{(df\_pop\_rating, }\FunctionTok{aes}\NormalTok{(}
  \AttributeTok{x =}\NormalTok{ media\_avaliacoes,}
  \AttributeTok{y =}\NormalTok{ media\_rating,}
  \AttributeTok{size =}\NormalTok{ n\_receitas,}
  \AttributeTok{label =}\NormalTok{ country}
\NormalTok{)) }\SpecialCharTok{+}
  \FunctionTok{geom\_point}\NormalTok{(}\AttributeTok{alpha =} \FloatTok{0.6}\NormalTok{, }\AttributeTok{color =} \StringTok{"\#CC79A7"}\NormalTok{) }\SpecialCharTok{+}
  \FunctionTok{geom\_text}\NormalTok{(}\AttributeTok{check\_overlap =} \ConstantTok{TRUE}\NormalTok{, }\AttributeTok{vjust =} \SpecialCharTok{{-}}\DecValTok{1}\NormalTok{, }\AttributeTok{size =} \DecValTok{3}\NormalTok{) }\SpecialCharTok{+}
  \FunctionTok{scale\_size\_continuous}\NormalTok{(}\AttributeTok{range =} \FunctionTok{c}\NormalTok{(}\DecValTok{2}\NormalTok{, }\DecValTok{10}\NormalTok{)) }\SpecialCharTok{+} \CommentTok{\# Ajusta o tamanho das bolhas}
  \FunctionTok{labs}\NormalTok{(}
    \AttributeTok{title =} \StringTok{"Popularidade x qualidade por país / cozinha"}\NormalTok{,}
    \AttributeTok{x =} \StringTok{"Média de avaliações por receita (Popularidade)"}\NormalTok{,}
    \AttributeTok{y =} \StringTok{"Média de nota (Qualidade {-} avg\_rating)"}\NormalTok{,}
    \AttributeTok{size =} \StringTok{"Número de receitas"}
\NormalTok{  ) }\SpecialCharTok{+}
  \FunctionTok{theme\_minimal}\NormalTok{()}
\end{Highlighting}
\end{Shaded}

\pandocbounded{\includegraphics[keepaspectratio]{Trabalho-Final---Elementos-de-Programação_files/figure-pdf/unnamed-chunk-11-1.pdf}}

\begin{Shaded}
\begin{Highlighting}[]
\CommentTok{\# Gráfico: tendência por ano}
\NormalTok{df\_year }\OtherTok{\textless{}{-}}\NormalTok{ cuisine\_year }\SpecialCharTok{|\textgreater{}}
  \FunctionTok{filter}\NormalTok{(}
\NormalTok{    year }\SpecialCharTok{\textgreater{}=}\NormalTok{ ANO\_MIN\_FIXO,}
\NormalTok{    year }\SpecialCharTok{\textless{}=}\NormalTok{ ANO\_MAX\_FIXO}
\NormalTok{  )}

\CommentTok{\# Seleciona os países para a tendência}
\ControlFlowTok{if}\NormalTok{ (PAIS\_TENDENCIA\_FIXO }\SpecialCharTok{==} \StringTok{"top"}\NormalTok{) \{}
\NormalTok{  paises\_sel }\OtherTok{\textless{}{-}}\NormalTok{ ranking\_paises}\SpecialCharTok{$}\NormalTok{country }\CommentTok{\# Usa o ranking fixo}
\NormalTok{  df\_tendencia }\OtherTok{\textless{}{-}}\NormalTok{ df\_year }\SpecialCharTok{|\textgreater{}}
    \FunctionTok{filter}\NormalTok{(country }\SpecialCharTok{\%in\%}\NormalTok{ paises\_sel)}
\NormalTok{\} }\ControlFlowTok{else}\NormalTok{ \{}
\NormalTok{  df\_tendencia }\OtherTok{\textless{}{-}}\NormalTok{ df\_year }\SpecialCharTok{|\textgreater{}}
    \FunctionTok{filter}\NormalTok{(country }\SpecialCharTok{==}\NormalTok{ PAIS\_TENDENCIA\_FIXO)}
\NormalTok{\}}

\FunctionTok{ggplot}\NormalTok{(df\_tendencia, }\FunctionTok{aes}\NormalTok{(}\AttributeTok{x =}\NormalTok{ year, }\AttributeTok{y =}\NormalTok{ total\_avaliacoes, }\AttributeTok{color =}\NormalTok{ country, }\AttributeTok{group =}\NormalTok{ country)) }\SpecialCharTok{+}
  \FunctionTok{geom\_line}\NormalTok{(}\AttributeTok{linewidth =} \DecValTok{1}\NormalTok{) }\SpecialCharTok{+}
  \FunctionTok{geom\_point}\NormalTok{() }\SpecialCharTok{+}
  \FunctionTok{labs}\NormalTok{(}
    \AttributeTok{title =} \FunctionTok{paste}\NormalTok{(}\StringTok{"Tendência de popularidade (Total de avaliações)"}\NormalTok{, }
                  \FunctionTok{ifelse}\NormalTok{(PAIS\_TENDENCIA\_FIXO }\SpecialCharTok{==} \StringTok{"top"}\NormalTok{, }\StringTok{"para os Top países"}\NormalTok{, }\FunctionTok{paste}\NormalTok{(}\StringTok{"para"}\NormalTok{, PAIS\_TENDENCIA\_FIXO))),}
    \AttributeTok{x =} \StringTok{"Ano de publicação"}\NormalTok{,}
    \AttributeTok{y =} \StringTok{"Total de avaliações no ano"}\NormalTok{,}
    \AttributeTok{color =} \StringTok{"País / cozinha"}
\NormalTok{  ) }\SpecialCharTok{+}
  \FunctionTok{theme\_minimal}\NormalTok{() }\SpecialCharTok{+}
  \FunctionTok{scale\_x\_continuous}\NormalTok{(}\AttributeTok{breaks =}\NormalTok{ scales}\SpecialCharTok{::}\FunctionTok{pretty\_breaks}\NormalTok{(}\AttributeTok{n =} \DecValTok{10}\NormalTok{))}
\end{Highlighting}
\end{Shaded}

\pandocbounded{\includegraphics[keepaspectratio]{Trabalho-Final---Elementos-de-Programação_files/figure-pdf/unnamed-chunk-12-1.pdf}}

\subsubsection{Análises e
Interpretações}\label{anuxe1lises-e-interpretauxe7uxf5es}

As interpretações centrais do estudo convergem na primazia da qualidade
intrínseca sobre o esforço extrínseco. A correlação nula entre o tempo e
os ingredientes versus a nota média implica que os usuários valorizam o
resultado final da receita sobre o custo de preparo, sugerindo que o
foco na otimização do sabor e na clareza da instrução é mais
determinante do que a complexidade.

Em contraste, a diferença significativa nas médias de avaliação entre
culinárias continentais sugere que as tradições gastronômicas
consolidadas (origem) são um forte preditor da qualidade percebida.

A distinção entre popularidade e qualidade, por sua vez, reforça a
necessidade de se utilizar métricas ponderadas ao ranquear culinárias.
Um alto volume de interesse (popularidade) pode não ser um substituto
direto para a excelência percebida (alta média de nota), e a dinâmica
temporal do conteúdo é um fator chave para o engajamento na plataforma.

\subsection{Conclusão}\label{sec-04}

O presente estudo demonstrou, através de análises estatísticas
rigorosas, que o esforço e a complexidade de uma receita não são fatores
determinantes para a sua avaliação final pelos usuários, com o
coeficiente de Pearson próximo de zero estabelecendo a independência
entre estas variáveis e a satisfação. Em contrapartida, a origem
geográfica da culinária influencia a avaliação média, sugerindo a alta
relevância da tradição. O estudo também enfatizou a distinção analítica
entre popularidade e qualidade, alertando contra a utilização de
métricas de volume como substitutas para a percepção de excelência. Em
síntese, a satisfação do usuário é primariamente guiada por fatores
intrínsecos de sabor e qualidade de execução, e não por custos de tempo
ou esforço.

\subsection*{Referências Bibliográficas}\label{sec-05}
\addcontentsline{toc}{subsection}{Referências Bibliográficas}

\phantomsection\label{refs}
\begin{CSLReferences}{0}{1}
\bibitem[\citeproctext]{ref-agresti_introduction_2019}
AGRESTI, Alan. \textbf{An introduction to categorical data analysis}.
Third edition ed. Hoboken, NJ: Wiley, 2019.

\bibitem[\citeproctext]{ref-noauthor_allrecipes_nodate}
\textbf{Allrecipes {\textbar} {Recipes}, {How}-{Tos}, {Videos} and
{More}}. \textbf{Allrecipes}, \emph{{[}S.d.{]}}. Disponível em:
\textless{}\url{https://www.allrecipes.com/}\textgreater. Acesso em: 13
nov. 2025

\bibitem[\citeproctext]{ref-hussain_innovative_2025}
HUSSAIN, Zahid \emph{et al.} (EDS.).
\textbf{\href{https://doi.org/10.4018/979-8-3693-8542-5}{Innovative
{Trends} {Shaping} {Food} {Marketing} and {Consumption}:}}
\emph{{[}S.l.{]}}: IGI Global, 2025.

\bibitem[\citeproctext]{ref-silva_da_costa_percepcao_2021}
SILVA DA COSTA, Ana Carolina; AMORIM, Maria Marta Amancio.
\href{https://doi.org/10.33448/rsd-v10i12.20461}{A percepção de
internautas sobre as receitas mais acessadas em mídia digital}.
\textbf{Research, Society and Development}, v. 10, n. 12, p.
e455101220461, set. 2021.

\bibitem[\citeproctext]{ref-trattner_estimating_2017}
TRATTNER, Christoph; ELSWEILER, David; HOWARD, Simon.
\href{https://doi.org/10.3389/fpubh.2017.00016}{Estimating the
{Healthiness} of {Internet} {Recipes}: {A} {Cross}-sectional {Study}}.
\textbf{Frontiers in Public Health}, v. 5, fev. 2017.

\bibitem[\citeproctext]{ref-wickham_r_2017}
WICKHAM, Hadley; GROLEMUND, Garrett. \textbf{R for data science: import,
tidy, transform, visualize, and model data}. Sebastopol, CA: O'Reilly
Media, Inc, 2017.

\end{CSLReferences}




\end{document}
